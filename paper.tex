% This is samplepaper.tex, a sample chapter demonstrating the
% LLNCS macro package for Springer Computer Science proceedings;
% Version 2.20 of 2017/10/04
%
\documentclass[runningheads]{llncs}
%
\usepackage{graphicx}
\usepackage[english,russian]{babel}
\usepackage[utf8]{inputenc}
%всякие настройки по желанию%
\usepackage[colorlinks=true,linkcolor=blue,unicode=true]{hyperref}
\usepackage{euscript}
\usepackage{supertabular}
\usepackage[pdftex]{graphicx}
\usepackage{wrapfig}
\usepackage{subfigure}
\usepackage{amsthm, amssymb, amsmath}
\usepackage{textcomp}
\usepackage{float}
\usepackage{listings}

\usepackage[noend]{algorithmic}
\usepackage[ruled]{algorithm}
\usepackage[]{algorithm2e}
\selectlanguage{russian}
% Used for displaying a sample figure. If possible, figure files should
% be included in EPS format.
%
% If you use the hyperref package, please uncomment the following line
% to display URLs in blue roman font according to Springer's eBook style:
% \renewcommand\UrlFont{\color{blue}\rmfamily}

\begin{document}
\begin{titlepage}
\begin{center}
% Upper part of the page
\textbf{\large Санкт-Петербургский государственный университет} \\[0.8cm]
\textbf{\large Математико-механический факультет} \\[6.0cm]

% Title
\textbf{\LARGE Погорелов Петр Глебович}\\[1.0cm]
\textbf{\Large Научно-исследовательская работа} \\[0.4cm]
 \textbf{\Large Ортогонализация набора данных с учетом применяемой решающей функции} \\[5.0cm]

%supervisor
\begin{flushright} \large
\textbf{Научный руководитель:} \\
\textbf{д. ф.-м. н., профессор Новиков Б.А.} \\
\textbf{Научный консультант:} \\
\textbf{к. ф.-м. н., доцент Кураленок И.Е.}
\end{flushright}
\vfill

% Bottom of the page
{\large \textbf{{Санкт-Петербург}}} \par
{\large \textbf{{2018}}}
\end{center}
\end{titlepage}
\newpage

%
\title{Ортогонализация Набора Данных с Учетом Применяемой Решающей Функции}
%
%\titlerunning{Abbreviated paper title}
% If the paper title is too long for the running head, you can set
% an abbreviated paper title here
%
\author{Погорелов П.Г.\inst{1}}
%
\authorrunning{Погорелов П.Г.}
\titlerunning{Ортогонализация набора данных}
% First names are abbreviated in the running head.
% If there are more than two authors, 'et al.' is used.
%
\institute{Санкт-Петербургский Государственный Университет, Санкт-Петербург, Россия\\
\email{peter.pogorelov@gmail.com}}
%
\maketitle              % typeset the header of the contribution
%
\begin{abstract}
В рамках настоящего исследования представлен результат работы над новым алгоритмом ортогонализации данных. Особенность данного алгоритма заключается в том, что в процессе декорреляции используется произвольная решающая функция. Это позволяет сформировать новый набор данных таким образом, что он оптимально подходит к конкретной модели классификации либо регрессии.

\keywords{машинное обучение  \and декорреляция набора данных \and ортогонализация набора данных.}
\end{abstract}
%
%
%
\section{Введение}
В процессе моделирования некоторой непрерывной $y \in R$ либо дискретной $y \in L = \{l_1, l_2, ..., l_k\}$ величины на основе реального (не синтетического) набора данных $X = (x_1, x_2, ... x_m)$, где $x_i$ может быть как вещественным, так и бинарным фактором, очень часто можно наблюдать так называемую проблему мультиколлинеарности (или эндогенности факторов). Проблема мультиколлинеарности сводится к наличию линейной связи между величинами, которая выражается в существовании их линейной комбинации (т.е., один фактор можно выразить через набор других). В большинстве математических моделей присутствие линейных комбинаций факторов приводит к проблемам, связанным с статистическими характеристиками оценок их коэффициентов: эффективности, состоятельности, несмещенности. В частности, если аналитически выразить коэффициенты линейной регрессии $\hat{b} = (X^TX)^{-1}X^Ty$, то можно обратить внимание, что элементы матрицы $(X^TX)^{-1}$ устремятся в бесконечность, если в данных будет присутствовать мультиколлинеарность. Это продиктовано тем, что ранг квадратной матрицы $X^TX$ будет меньше числа ее столбцов, что, следовательно выразится в равенстве определителя $det(X^TX)$, либо его стремлении к $0$.
\par
В рамках теории информации, проблему мультиколлинеарности можно интерпретировать как наличие взаимной информации в отдельных факторах друг о друге. Соответственно — один из подходов к устранению мультиколлинеарности — ортогонализация (или декорреляция) набора данных. Для решения задачи ортогонализации данных существует множество подходов, которые можно разделить на две крупные категории,
соответственно, линейные и нелинейные методы декорреляции. Также можно отдельно упомянуть класс методов декорреляции данных, называемый whitening-трансформация. Её суть заключается в том, чтобы сделать ковариационную матрицу декоррелированных факторов в новом базисе — единичной. \par
Цель настоящей работы — разработка алгоритма, позволяющего выполнить ортогонализацию (декорреляцию) набора данных с использованием произвольной решающей функции. Для достижения этой цели были поставлены задачи:
\begin{enumerate}
  \item выполнить обзор наиболее популярных линейных и нелинейных алгоритмов ортогонализации данных,
  \item формализовать новый алгоритм ортогонализации набора данных,
  \item формализовать задачи, решаемые в рамках исследования, для которых может потребоваться выполнить ортогонализацию данных,
  \item сравнить на разных наборах данных результат работы данных алгоритмов на произвольных моделях регрессии, классификации,
  \item оценить его качество на произвольных моделях классификации.
\end{enumerate}
В рамках последующих разделов представленного исследования будут рассмотрены все поставленные задачи, а в заключительной части сформулированы выводы касательно итогового результата работы.
\section*{Обзор существующих подходов}

В настоящий момент существует множество алгоритмов ортогонализации (декорреляции) данных успешно применяемых в различных прикладных задачах, которые можно разделить на две крупные категории. Первая категория — линейные алгоритмы, вторая, соответственно — алгоритмы нелинейные. В первой категории особенно хочется выделить:
\begin{enumerate}
  \item ZCA-Whitening
  \item Principal Component Analysis
  \item Independent Component Analysis \cite{fastica}
\end{enumerate}
Во второй категории наибольший интерес представляют следующие методы:
\begin{enumerate}
  \item Isomap \cite{isomap}
  \item Multi-Dimensional Scaling \cite{mds}
  \item Locally Linear Embedding \cite{locallyle}
  \item t-SNE  \cite{tsne}
\end{enumerate}
В следующих подразделах будет детально рассмотрена часть из представленных выше алгоритмов.
\addcontentsline{toc}{section}{Обзор существующих подходов}
\subsection{Principal Component Analysis}
\hspace{0.4cm}
Метод главных компонент (PCA) можно назвать одним из важнейших прикладных инструментов математического моделирования, использующих аппарат линейной алгебры. Благодаря своей простоте и интуитивной интерпретации, метод главных компонент пользуется широкой популярностью специалистами различных областей, в круг задач которых входит аналитика над данными. Среди таких областей можно выделить: нейрофизиологию, биоинформатику, компьютерную графику, физику и др. \par
Данный непараметрический подход к декорреляции используется для извлечения полезной информации в зашумленных наборах данных. Благодаря тому факту, что метод главных компонент практически не содержит гиперпараметров (кроме количества выделяемых компонент), даже человек без математического образования может провести редукцию размерности комплексоного пространства данных с целью построения визуализации, подготовки входных параметров к некоторой математической модели, выявления скрытых закономерностей в данных.

Особенность метода главных компонент заклчается в подходе к выделению такого нового базиса матрицы факторов, что факторы в данном базисе становятся ортогональными (линейно независимыми). То есть, будет иметь место нулевая корреляция Пирсона между ними. Еще одна интересная особенность PCA заключается в том, что выделенные компоненты также будут отранжированы по степени выриации данных вдоль их проекции на компоненту. Иными словами — компоненты будут упорядочены по степени значимости (вклада) каждой из компонент в имеющийся набор данных.

Далее представлен алгоритм построения нового базиса для матрицы факторов, который используется в методе главных компонент,
\begin{enumerate}
  \item сначала рассчитывается ковариационная матрица по набору данных,
  \item для ковариационной матрицы рассчитываются соответствующие ей собственные вектора и собственные значения,
  \item выполняется сортировка собственных векторов в порядке убывания соответсвующих им собственных чисел,
  \item первые k собственных векторов рассматриваются как новый базис, выделяющий k независимых факторов,
  \item с использованием линейного преобразования, построенного на предыдущем шаге, оригинальные данные преобразуются в k-мерные точки пространства нового базиса.
\end{enumerate}

Таким образом, можно обозначить наиболее важные задачи, решаемые в рамках инструменратия, предлагаемого аппаратом метода главных компонент. Соответственно, первая - поиск линейно независимых подпространств внутри многомерного облака данных, на которые можно спроецировать данные с минимальными потерями. Вторая - предоставление возможности выполнить реконструкцию оригинальных измерений, то есть, опять же, потери должны быть минимальны. Суть метода заключается в том, чтобы итеративно находить такие проекции внутри облака данных, что вариация вдоль проекции будет максимальна, а проекции - ортогональны друг другу. \par

Формализовать метод главных компонент можно следующим образом. Пусть имеется некоторой набор данных ${X \in \mathbb{R}^m}$, где каждый из факторов является либо непрерывной случайной величиной $x_i \in R$, либо бинарной случайной величиной $x_i \in \{0, 1\}$. Ковариация случайного вектора $x$ рассчитывается по формуле, $$Cov(x_i, x_j) = \frac{1}{n} \sum_{t=1}^n (x_i(t) - E[x_i])(x_j(t) - E[x_j]).$$ \par 

В случае, если факторы ноль-центрированы ($E[x] = (0, 0, ...,0)$), то представленное выше выражение можно упростить до следующей векторно - матричной формы:  $Cov(X) = X^TX.$ Далее, необходимо найти такое линейное преобразование $\hat{X} = PX$ что $Cov(\hat{X}) = \hat{\sum} = I$. То есть, ковариационная матрица набора данных из исходного пространства ($X$) в новом базисе $P$ ($\hat{X} = PX$) становится единичной. Ковариацию набора данных в новом базисе можно выразить следующим образом,
$\hat{\sum} = \hat{X}^{T}\hat{X} = (PX)^T(PX) = P^TX^TXP = P^{T}\sum P$. Очевидно, что если взять за $P$ матрицу, построенную из собственных векторов исходной ковариационной матрицы $\sum$, то $\sum P = \Lambda P$, где  $\Lambda$ - диагональная матрица, на главной диагонали которой лежат собственные числа исходной ковариационной матрицы. Данное выражение напрямую вытекает из свойств собственных чисел и собственных векторов произвольной матрицы. То есть, если взять квадратную матрицу $A$ для которой $V$ - матрица собственных векторов, $\lambda$ - вектор собственных чисел, то будет справедливо выражение выражение $AV = \lambda IV$. Таким образом, становится очевидной справедливость выражения $\hat{\sum} = P^T P \Lambda = I \Lambda$, в результате которого ковариационная матрица в новом базисе стала диагональной, и, следовательно, была выполнена декорреляция факторов, устранены линейные зависимости между ними. Теперь, если выполнить деление каждой из получившихся в результате рассмотренного преобразования компонент на квадратный корень соответствующего ей собственного числа, то данные будут сферизованы (ковариационная матрица станет единичной).
\subsection{Independent Component Analysis}
\hspace{0.4cm}
Метод независимых компонент (ICA) - еще один из линейных подходов к ортогонализации. Его принципиальное отличие от <<классического>> метода главных компонент заключается в том, в процессе <<обучения>> данного алгоритма в облаке данных ищутся такие проекции, где мера отличия от нормального распределения — максимальна. Метод анализа независимых компонент базируется на технике Projection Pursuit (гонка за проекцией) \cite{projpurs}, суть которой которой заключается в том, чтобы найти некоторые <<интересные>> проекции в многомерном пространстве данных.\par Формально, ICA можно рассматривать как частный случай реализации алгоритма Projection Pursuit. ICA позволяет (в прочем, наряду с PCA) выделять независимые компоненты ($s$) на основе имеющегося набора данных ($x$) такие, что что $|s| \leq |p|$. То есть, число новых выделенных компонент может быть как меньше изначального числа представленных факторов, так и быть равным ему. Исходя из предположения, что некоторые измерения исходного пространства данных, не входящих в описываемое выделенными независимыми компонентами подпространство, заполнено белым шумом (gaussian noise), можно сформировать предпосылку о необходимости поиска такие проекции внутри исходного пространства, распределение данных вдоль которых максимально отличается от нормального. Соответственно, после того как все <<не-гауссовские>> проекции были найдены, можно считать, что все независимые компоненты были выделены. \par
Для того, чтобы интегрировать понятие <<не-гауссовости>> в метод анализа независимых компонент, необходимо формализовать некоторую меру, позволяющую определить, насколько сильно некоторая случайная величина $y$ отклоняется от нормального распределения. Для простоты предлагается ввести предположение о ноль-центрировании $y$ ($E[y] = 0$) и о её единичной дисперсии ($\sigma_y = 1$).\par
В качестве классической меры <<не-гауссовости>> часто рассматривают величину эмпирического коэффициента эксцесса (нормализованная версия момента четвертого порядка $E[y^4]$), который можно выразить следующим образом $$kurt(y) = E[y^4] - 3(E[y^2])^2.$$
В случае, если величина $y$ распределена нормально, коэффициент эксцесса будет равен нулю, в то время как для большинсва случайных величин из других распределений (не гауссовых) величина данного коэффициена будет отлична от нуля.\par
Еще одна важная мера <<не-гауссовости>> задается с помощью негэнтропии (negentropy). Для начала необходимо дать определение понятия энтропия. Энтропия представляет собой один из базовых концептов теории информации и может быть интерпретирована как объем информации, который может содержаться в случайной величине. Чем более <<случайна>> (непредсказуема) величина, тем выше её энтропия. Более формально, энтропия представляет собой оптимальную длинну строки, которая необходима, чтобы закодировать значения, принимаемые случайной величиной. В дискретном случае энтропию можно выразить следующим образом, $$H(Y) = -\sum_i P_y(y_i) log(P_y(y_i)).$$\par
В случае, если величина $y$ принадлежит к некоторому непрерывному распределению, то величину энтропии можно задать следующим образом, $$H(y) = -\int f(y)log(f(y))dy.$$ \par
Пользуясь тем фактом, что случайная величина, порожденная нормальным распределением, обладает самой высокой энтропией среди всех случайных величин с тем же значением дисперсии, можно формализовать меру <<не-гауссовости>> как значение отклонения энтропии некоторой случайной величины от энтропии нормального распределения. Эту меру принято называть негэнтропией, которая задается следующим образом, $$J(y) = H_{gauss} - H(y).$$ \par
Поскольку рассчет настоящего значения негэнтропии — весьма трудоемкая задача в смысле затрачиваемых вычислительных ресурсов, принято использовать различные апроксимации. В частности, классический метод, разработанный Jones с соавторами (1987) использует аппроксимацию негэнтропии с использованием моментов высших порядков, $$J(y) \approx \frac{1}{12}E\{y^3\}^2 + \frac{1}{48}kurt\{y\}^2.$$ Важно отметить, что данную аппроксимацию нельзя назвать робастной в отношении коэффициента эксцесса, что затрудняет её применение в контексте обучения ICA модели. \par
Авторы публикации \cite{fastica} предлагают использовать в качестве аппроксимации негэнтропии следующую функциональную форму, которая содержит некоторую функцию $G$, которая будет формализована далее
$$J(y) \propto (E[G(y)] - E[G(v)])^2,$$
где $v$ представляет собой ноль-центрированную случайную величину из гауссовского распределения с единичной дисперсией (стандартизированную). Аналогичным образом подразумевается, что величина $y$ тоже должна быть ноль-центрирована и иметь дисперсию равную единице. Очевидно, что $J(\cdot)$ представляет собой обобщение аппроксимации негэнтропии, базирующееся на использовании моментов, которая была рассмотрена ранее. Если взять такую функциональную форму $G$, которая не вызовет слишком резкий рост аппроксимированной величины негэнтропии $J$, то оценка негэнтропии будет считаться относительно устойчивой (робастой). В частности, авторы публикации предлагают использовать следующие функциональные формы $G,$
$$G_1(u) = \frac{1}{a_1}\log \cosh _{a1}(u), G_2(u) = -exp(-u^2 / 2),$$
где $1 \leq a_1 \leq 2$ - некоторая константа.\par

Один из самых популярных подходов для оценки параметров ICA модели — использование метода максимального правдоподобия, который имеет тесную связь с принципом infomax. Формализовать данный подход можно следующим образом. Пусть $W = (w_1, w_2, ..., w_k)^T = A^{-1}$, где $A$ — линейное ортогональное преобразование, выделяющее независимые компоненты из набора данных $X = (x_1, x_2, ..., x_m)$. Лог-правдоподобие будет иметь вид, $$L = \sum_{i=1}^{N}\sum_{j=1}^{k}\log{f_j(w_j^{T}x(i))} + T\log(det\ W),$$
где $f_i$ — плотность распределения независимой компоненты $s_i$ (предполагается, что плотность известна), а $x(i)$ — наблюдения случайной величины $x$. Правый терм выражения, $log |det W|$ — напрямую следует из правила линейного преобразования случайных величин и их плотностей. Известно, что для любого случайного вектора $x$ с плотностью $p_x$ и для любой матрицы $W$ плотность $y = Wx$ выражается как $p_x(Wx)|det W|$. \par
Далее предлагается формализовать алгоритм FastICA, который на текущий момент является самой популярной реализацией метода ICA для выделения независимых компонент в наборе данных. В своей базовой форме алгоритм можно выразить следующим образом,

\begin{enumerate}
  \item выполняется предварительная инициализация вектора весов $w$, например, значения могут быть сгенерированы из стандартизированного нормального распределения,
  \item далее рассчитывается <<обновление>> вектора весов $w^+ = E[xg(w^Tx)] - E[g'(w^Tx)]w$,
  \item затем выполняется перерасчет вектора весов $w = w^+/\|w^+\|$,
  \item шаги 2, 3 повторяются до тех пор, пока выражение не сошлось, то есть, пока $\|w\|_2 > \epsilon$, где $\epsilon$ - некоторая константа, например, $1e^-10$.
\end{enumerate}

Далее предлагается рассмотреть вывод выражения для $w^+$. Пользуясь следствием теоремы Кунна-Таккера, можно утверждать что оптимальное значение для $E[G(w^Tx)]$ при заданном ограничении $E[(w^Tx)^2] = \|w\|^2 = 1$ достигается в точке, где выполняется следующее условие,
$$E[xg(w^Tx)] - \beta w = 0.$$ \par
Данное выражение можно попытаться оптимизировать с помощью метода Ньютона. В частности, якобиан функции, используемый в контексте ограничения выше можно выразить следующим образом,
$$JF(w) = E[xx^Tg'(w^Tx)] - \beta I.$$ \par
Далее, исходя из предположения, что данные были предварительно сферизованы (была применена некоторая whitening-трансформация, например, PCA, либо ZCA), можно ввести следующую аппроксимацию, $$E[xx^Tg'(w^Tx)] \approx E[xx^T]E[g'(w^Tx)] = E[g'(w^Tx)]I.$$ \par
Таким образом, аппроксимированная итерация метода Ньютона для оценки коэффициенов вектора $w$ примет следующий вид, 
$$w^+ = w - [E[xg(w^Tx)] - \beta w] / [E[g'(w^Tx)] - \beta].$$

\subsection{Locally Linear Embedding}
\hspace{0.4cm}
Локально-линейный эмбеддинг (Locality Linear Embedding или LLE) — представляет собой метод для поиска координат в пространстве с низкой размерностью для набора данных, лежащего на некотором нелинейном многообразии, построенном на высоко-размерном пространстве. Идея метода заключается в том, чтобы индивидуально, для каждой отдельной точки, выполнять редукцию размерности. Предпосылка к использованию данного подхода заключается в том, что по утверждениям автора метода, любое нелинейное многообразие имеет локально-линейную структуру, аналогично интерпретации обыкновернной производной гладкой функции. Далее, после того, как размерность каждой из точек была редуцирована, авторы предлагают расположить точки в новом пространстве таким образом, чтобы некоторая мера <<несоответствия>> их локальной структуры была минимальна. \par

Сама процедура LLE преобразования выполняется в три шага. Сначала определяются ближайшие соседи для каждой из точек в представленном наборе данных. Затем рассчитываются веса для их линейной аппроксимации посредством их соседей. Заключительный этап алгоритма — поиск низкоразмерных координат, которые реконструируются полученными на прошлом шаге весами наиболее эффективным образом. \par
Формализовать алгоритм LLE можно следующим образом. Пусть имеется матрица $X = (x_1, x_2, ..., x_m)$, где каждый из $m$ факторов описывается $n$ наблюдениями. Задается размерность итогового низкоразмерного подпространства $q < m$ и число ближайших соседей $k$ такое, что $k \geq q + 1$. Результатом работы алгоритма будет матрица $\hat{X} = (\hat{x}_1, \hat{x}_2, ..., \hat{x}_q, )$. Механизм работы LLE преобразования формально разбивается на следующие шаги,
\begin{enumerate}
  \item для каждой точки $X_i$ рассчитываются $k$ ближайших соседей,
  \item Рассчитывается матрица весов $W$, которая минимизирует остаточную сумму квадратов для реконструкции точки $X_i$ по её соседям
  $$RSS(W) = \sum_{i=1}^n\|X_i - \sum_{j \neq i}W_{ij}X_j\|^2,$$
  где $W_{ij} = 0$ для тех $X_j$, которые не являются соседними для $X_i$ точками. Помимо всего прочего, для каждого $i$ должно выполняться $\sum_j W_{ij} = 1$,
  \item рассчитываются координаты $\hat{X}$, минимизирующие ошибку реконструкции точек $X$ в низкоуровневом пространстве с использованием матрицы весов $W$, полученной на предыдущем шаге,
  $$\Phi(\hat{X}) = \sum_{i=1}^n\|\hat{X}_i - \sum_{i \neq j} W_{ij}\hat{X}_j\|^2,$$
  где $\sum_i\hat{X}_{ij} = 0$ для каждой $j$, а $\hat{X}^T\hat{X} = I$.
\end{enumerate}
\subsection{t-Distributed Stochastic Neighbor Embedding}
\hspace{0.4cm}
t-SNE представляет собой алгоритм нелинейного сжатия размерности, широко используемый в задачах исследования структуры многомерных данных. В большинстве случаев, данный алгоритм используется для проекции многомерных данных на две (и более) оси координат с целью упрощения визуального анализа данных. На практике, преобразованные при помощи t-SNE факторы не используется в качестве входных данных для статистических моделей и моделей машинного обучения в силу высокой вычислительной сложности данного алгоритма. Фактически, сложность зависит от реализации, но во всех случаях она — полиномиальная в зависимости от числа наблюдений и количества факторов. \par

Для начала можно сформулировать некоторые предпосылки к использованию t-SNE, которые выделяют его на фоне линейных методов сжатия размерности данных, в том числе, метода главных компонент (PCA) и анализа независимых компонент (ICA). Следствием того, что PCA и ICA представляют собой линейные алгоритмы декорреляции, является их неспособность к интерпретации комплексных полиномиальных отношений между факторами. С другой стороны, t-SNE — базируется на вероятностных распределениях со случайным блужданием по графам ближайших соседей, с помощью чего данный алгоритм проводит анализ структуры набора данных, и, теоретически, способен выделять более сложные взаимосвязи, нежели линейные.\par

Самая главная проблема, связанная с применением линейных алгоритмов сжатия размерности заключается в том, что их механизм работы сводится к расположению <<наименее близких>> друг к другу точек как можно дальше друг от друга в пространстве, размерность которого меньше исходного. Однако следует обратить внимание, что для того, чтобы представить проекцию набора данных из пространства высокой размерности в нелинейном многообразии низкой размерности, необходимо сфокусироваться на анализе схожести точек, выступающих друг для друга ближайшими соседями. Линейные алгоритмы сжатия размерности пространства, фактически, не способны на это. \par

Особенность t-SNE в сравнении с другими алгоритмами сжатия нелинейных многобразий высокой размерности (в т.ч. LLE) заключается в том, что данный алгоритм способен восстанавливать не только локальную, но и глобальную структуру данных. Подходы к восстановлению локальной структуры ищут наиболее близкие друг к другу точки на многообразии и размещают их близко к друг другу в пространстве мешньшей размерности. Подходы к восстановлению глобальной структуры напротив, пытаются сохранить глобальную геометрию пространства на всех машстабах, то есть, располагают близкие друг к другу на многообразии точки близко друг к другу, а находящиеся далеко друг от друга — как можно дальше друг от друга. \par

Далее будут рассмотрены детали работы алгоритма t-SNE. Для начала, необходимо рассмотреть алгоритм-прародитель t-SNE — SNE или Stochastic Neighbor Embedding. Досновно его название можно перевести как "стохастический метод построения сжатого представления данных с использованием метода ближайших соседей". Данный алгоритм начинает свою работу с конвертации евклидовых расстояний между точками набора данных в условные вероятности, которые можно интерпретировать как меру схожести. Мера схожести точки $x_i$ и точки $x_j$ выражается в виде условной вероятности $p(x_i | x_j)$ (далее используется обозначение $p_{i|j}$). Точка $x_i$ <<интерпретирует>> точку $x_j$ как своего соседа, если ее отклонение от $x_i$, которое выступает в качестве математического ожидания некоторой гауссианы, входит в допустимые рамки, ограниченные некоторой пропорцией (задающейся числом ближайших соседей). \par

Для точек, которые располагаются сравнительно близко друг к другу, величина $p_{i|j}$ — относительно велика, в то время как для точек, которые находятся далеко друг от друга - величина $p_{i|j}$ — стремится к нулю. Математически величину $p_{i|j}$ можно выразить следующим образом, $$p_{i|j} = \frac{exp(-\|x_i - x_j\|^2 / 2\sigma_i^2)}{\sum_{k \neq i} exp(-\|x_i - x_k\|^2 / 2\sigma_i^2)},$$
где $\sigma_i$ — среднеквадратичное отклонение гауссианы, центрированной в точке $x_i$. \par

Введем обозначение $y$, которое можно интерпретировать как проекцию точки $x$ в пространстве меньшей размерности. В таком случае, можно ввести аналог вероятности $p_{i|j}$ и обозначить его как $q_{i|j}$, $$q_{i|j} = \frac{exp(-\|y_i - yx_j\|^2 / 2\sigma_i^2)}{\sum_{k \neq i} exp(-\|y_i - y_k\|^2 / 2\sigma_i^2)}.$$

Проще говоря, величина $p_{i|j}$ отражает близость точек в пространстве высокой размерности, а $q_{i|j}$ — в пространстве низкой размерности. Интуитивно, для того, чтобы добиться близости точек в пространствах разной размерности, данные условные вероятности должны быть равны друг другу. Алгоритм SNE как раз базируется на этих предпосылках, пытаясь минимизировать разницу между условными вероятностями. \par

В алгоритме t-SNE для минимизации разницы между условными вероятностями используется мера из теории информации — дивергенция Кульбака-Лейблера (Kullback—Leibler Divergence). Она рассчитывается по формуле $$KL(p \| q) = H(p) - H(p, q) = \sum_i p_i \log{q_i}.$$ Соответственно, в алгоритме t-SNE минимизируется следующий функционал,
$$KL(p \| q) = \sum_{i\neq j} \log\frac{p_{i|j}}{q_{i|j}} \rightarrow min_y.$$

Следует отметить, что t-SNE, как и LLE не подразумевают введение некоторого нелинейного преобразования на основе существующего набора данных, при помощи которого можно выполнить ортогонализацию факторов. Ортогонализация выполняется только по текущим данным, и в случае введения новой точки (наблюдения), все рассчеты потребуется производить заново.


\section*{Задачи классификации и регрессии}
\addcontentsline{toc}{section}{Задачи классификации и регрессии}
\hspace{0.4cm}
Задачи классификации и регрессии — одни из самых распространенных проблем, решаемых с помощью аппарата прикладной статистики либо, как на текущий момент принято выражаться, <<машинного обучения>>. Задача классификации представляет собой подраздел группы методов машинного обучения с учителем, суть которой заключается в том, чтобы <<обучить>> алгоритм выполнять автоматическое определение категории объекта, к которой он относится с наиболее высокой вероятностью. Обучение выполняется с использованием размеченной выборки, то есть, некоторому объекту $x \in R^m$ ставится в соответствие бинарная $y \in B$, либо дискретная категория (метка) $y \in L$, где $L = \{l_1, l_2, ..., l_k\}$ — множество некоторых дискретных величин (категорий), а $B = \{0, 1\}$ — множество бинарных величин, принимающих только два значения. Соответственно, цель <<обучения>> — поиск некоторой функциональной формы $f \in F$ и оценка ее коэффицинтов с целью минимизации шанса совершить ошибку классификации $f(x_i) \neq y_i$. Алгоритмические подходы решения проблемы классификации можно выделить в две большие группы — модели бинарной классификации и методы множественной классификации. В дальнейших подразделах данной главы будет представлен их краткий обзор. Задача линейной регрессии аналогична проблеме классификации во многих аспектах, отличие заключается в том, что моделируемая величина — непрерывна ($y \in R$), в остальном постановка проблемы ничем принципиально не отличается от случая классификации.

\subsection{Задача классификации}
Задача классификации — одна из наиболее популярных проблем, решаемых в контексте машинного обучения с учителем. Самый распространенный вид классификации — бинарная проблема, где моделируемая величина принимает только два уникальных значения значения (1 и 0, <<ложь>> и <<истина>>, <<да>> и <<нет>> и так далее). Формализовать бинарную классификацию можно следующим образом. Пусть имеется некоторый набор данных (чаще всего употребляется термин <<обучающая выборка>>) $X = \{x_i\}_{i=1}^n$, где $n$ — объем выборки. В соответствие каждому индивидуальному объекту $x_i$ поставлена метка $y \in \{0, 1\}$. Также предполагается, что каждый объект $x_i$, каждая метка $y_i$ и каждое наблюдение $(x_i, y_i)$ независимо сгенерированы из неизвестного распределения данных $P(x, y)$. Предположим, что существует некоторый генеративный процесс $P(y | x)$, который можно однозначно выразить в следующем виде,
$$y = f(x) + \epsilon,$$
где $\epsilon$ — некоторая случайная компонента (ошибка), а $f(\cdot)$ — функциональная зависимость между объектом выборки и соответствующей ему меткой. Соответственно, задача машинного обучения — подобрать функциональную форму $\hat{f} \in F$ аппроксимирующую реальную функциональную зависимость $f$ с минимальной ошибкой. \par
При работе с моделями машинного обучения, предназначенными для категоризации наблюдений выборки только на два подмножества, возникает потребность в использовании некоторого мета-алгоритма, позволяющего расширить функциональность исходных моделей для включения дополнительного набора классов. На текущий момент существует множество подобных мета-алгоритмов, наиболее популярные из них разобраны в работе Daniely A. \cite{multiclass} с соавторами. \par
One versus Rest или OvR (один против всех) — наиболее очевидный способ реализации “мультиклассовости”. Производится обучение $m$ бинарных классификаторов $f_i$ (где $i \in \{1, 2, ..., m\}$) разделяющих выборку между одним из классов $l_i \in L= \{l_1, l_2, ..., l_m\}$ и оставшимися $m - 1$ классами. На этапе предсказания, для каждого документа $x$ выбирается такой класс $\hat{l_x}$, что степень уверенности соответствующей модели $f_{\hat{l_x}}$ — максимальна,
$$\hat{l_x} = arg max_{\l \in L} f_l(x).$$
Под степенью уверенности классификатора можно понимать разные величины. Например, если речь идет о машине опорных векторов, то это – расстояние от классифицируемого объекта до разделяющей гиперплоскости. В логистической регрессии за данную величину можно принять предсказанную вероятность. В рамках данного исследования все модели бинарной классификации, функционал которых необходимо экстраполировать на множество классов, используют мета-алгоритм <<один против всех>>. \par
\subsection{Задача регрессии}
Регрессионный анализ — это широкая область статистического моделирования, занимающаяся исследованием взаимосвязей между случайными величинами. Одна из задач, решаемая в рамках регрессионного анализа — моделирование так называемой целевой величины. То есть, подразумевается, что одна из величин фиксируется как целевая, а остальные рассматриваются как независимые (регрессоры, факторы), с помощью которых можно предсказать (смоделировать) целевую. Если удалось однозначно установить, что связь между целевой величиной и экзогенными факторами — линейная, то задача сводится к решению проблемы линейной регрессии. В таком случае имеет место следующая функциональная зависимость между целевой величиной и факторами,
$$y_i = \beta_0 + \sum_{j=1}^m \beta_j x_{i, j} + \epsilon,$$
где $y_i$ — i-е наблюдение моделируемой величины, $x_{i, j}$ — j-я компонента i-го наблюдения экзогенной величины, $\beta_j$ — некоторая константа, величину которой необходимо установить в рамках решения задачи регрессии, а $\epsilon$ — случайный шум (и не учтенные факторы). Напротив, если величины в модели связаны между собой не линейно, то могут потребоваться дополнительные исследования, требующие глубокого понимания предметной области, чтобы установить эту связь.

\section*{Алгоритм подбора оптимального подмножества данных}
\addcontentsline{toc}{section}{Алгоритм подбора оптимального подмножества данных}
\hspace{0.4cm}
\subsection{Цель и задача алгоритма}
Пусть имеется некоторое множество независимых случайных величин $X = \{x_i\}_m$, принадлежащих к разным вероятностным распределениям. Требуется сформировать упорядоченное подмножество $\tilde{X} = \{x_j\ \|\ h(x_{j}) > h(x_{j+1})\}_m$, отсортированное по величине кросс-энтропии между переменной $x_j$ и ее предсказанным значением $\hat{x}_j$, полученным с помощью решающей функции $f \in F$, где $F$ — множество решающих функций. Функция $h(\cdot) \in H$ используется для ранжирования величин, $H$ — множество всех возможных подходов для ранжирования. В контексте данного исследования подразумевается использование кросс-энтропии между действительным значением величины и ее предсказанным значением в качестве меры ранжирования.

\subsection{Описание алгоритма}
Рассмотрим постановку задачи регрессии в контексте данного исследования. Под проблемой регрессии подразмевается моделирование условномого математического ожидания некоторой случайной величины (в дальнейшем предлагается использовать обозначение $y$) от набора других случайных величин, которые принято называть факторами (в дальнейшем предлагается использовать обозначение $X = (x_1, x_2, ... x_m)$. То есть, требуется установить некоторую функциональную форму $f(X)$, такую что $f(X) = E(y|X)$, где $f \in F$ некоторая функция-регрессор из множества решающих функций.
\par
В рамках исследования предлагается свести задачу регрессии к задаче множественной классификации. Это продиктовано тем, что рассчет величины кросс-энтропии между непрерывными величинами требует применения некоторых численных методов для оценки плотности переменных, между которыми рассчитывается данная мера. В частности, сложность рассчета ядерного сглаживания составляет $O(kn)$ \cite{raykar2010fast}, в то время как рассчет эмпирического распределения дискретной случайной величины - $O(n)$. Так же важно отметить, что дихтомизация непрерывных факторов устраняет такие проблемы выборки, как выбросы. Для решения задачи дихтомизации был выбран метод разбиения по медиане, рассматриваемый в работе R. MacCallum с соавторами \cite{maccallum2002practice}. В данной работе также рассматриваются как положительные, так и негативные эффекты, которые процесс дихтомизации выборки оказывает на качество классификации.
\par
В результате дихтомизации данных множества непрерывных случайных величин $X_R = \{{x_{R}}_i\}_{m_{R}}$, формируется новое множество $X_D = \{{x_{d}}_i\}_{m_{d}}$, такое, что ${x_{d}}_i \in \{0, 1\}$, а ${m_{R}} < {m_{d}}$. В рамках данного исследования предлагается производить дихтомизацию величин методом бисекции по медиане на 32 части. Таким образом, число новых бинарных величин ${m_{d}}$ будет равно $32{m_{R}}$. Код, выполняющий данную процедуру представлен в листинге на языке программирования Python. В данном случае переменная $x\_bins$ представляет собой массив новых, дихтомизированных величин из набора $X$.
\begin{lstlisting}
x_bins = [np.sort(X)]
for i in range(int(iterations)):
   new_bins = list()
   for j in x_bins:
      index = j.shape[0] // 2
      new_bins += [j[:index], j[index:]]
   x_bins = new_bins
\end{lstlisting}
\par
Далее к рассмотрению предлагается описание алгоритма подбора оптимального упорядоченного множества факторов. Пусть имеется набор вещественных факторов $X = (x_1, x_2, ..., x_m)$ и его дихтомизированное представление $X_D = ({x_D}_1, {x_D}_2, ..., {x_D}_m)$, где ${x_D}_i = ({{x_d}_i}_1, {{x_d}_i}_2, ..., {{x_d}_i}_{32})$. 

\begin{algorithm}[H]
 \KwData{$\tilde{X} \gets \{\varnothing\}, I \gets \{\varnothing\}$}
 \KwResult{Упорядоченное по значимости множество индексов факторов ($I$); упорядоченное по значимости множество ортогонализованных факторов ($\tilde{X}$)}
 $\tilde{X} \gets \tilde{X}\cup {x_D}_r; I \gets \{r\};\ \ 1 \leq r \leq m$ \par
 \While{$\|\tilde{X}\| \leq m$}{
     $h \gets \{\varnothing\}$\par
     $\hat{X} \gets \{\varnothing\}$\par
     \For{$x_D \in X_D \setminus \tilde{X}$}{
      $f \gets arg\max_{f \in F} L_F(x_D | \tilde{X})$\par
      $\hat{x_D} \gets x_D \oplus f(\tilde{X})$\par
      $h \gets h \cup H(\hat{x_D}, x_D)$\par
      $\hat{X} \gets \hat{X} \cup \hat{x_D}$
     }
     $\tilde{X} \gets \tilde{X} \cup \hat{X}[arg\max{h}]$\par
     $I \gets I \cup \{arg\max{h}\}$
   }
 \caption{Поиск упорядоченного подмножества факторов}
\end{algorithm}

Далее предлагается подробно рассмотреть представленный выше алгоритм. На этапе инициализации формируется пустое упорядоченное множество ортогонализованных факторов  $\tilde{X} \gets \{\varnothing\}$, куда добавляется один какой-нибудь случайный фактор ${x_D}_r$, $1 \leq r \leq m$. Аналогичным образом формируется пустое упорядоченное множество индексов, куда заносится индекс $r$. Далее, до тех пор, пока выполняется условие $\|\tilde{X}\| \neq \|X_D\|$ ($\|\tilde{X}\| \leq m$), реализуются следующие шаги,
\begin{enumerate}
  \item выполняется оценка коэффициентов решающей функции последовательно для каждого их элементов $x_D \in X_D$  на основе данных $\tilde{X}$, то есть, подбирается функциональная форма $\hat{f} \approx E[x_D | \tilde{X}]$. Рассчитывается величина $\hat{x_D}$, представляющая собой значение операции исключающего или между действительным значением величины $x_D$ и ее оценкой $\hat{f}(\tilde{X})$ при заданном $\tilde{X}$,
  \item рассчитывается величина кросс-энтропии $H(\cdot, \cdot)$ между результатом операции XOR и действительным значением,  $h' = H(\hat{x_D}, x_D)$.
  \item в множество $\tilde{X}$ добавляется величина $\hat{x_D}$ с максимальной величиной рассчитанной кросс-энтропии, а соответствующий ей индекс заносится в множество $I$.
\end{enumerate}
Аналогичным образом, множество $\tilde{X}$ рассчитывается для всех начальных $r \in \{1, 2, ..., m\}$. Обозначим множество $\tilde{X}$, построенное на некоторой начальной $r$ как $\tilde{X}_r$. Тогда, в результате последовательного выполнения алгоритма для всех уникальных значений параметра $r$ будет сформировано множество $\tilde{X}_R = \{\tilde{X}_r\}_m$, элементы которого - различные подмножества ортогонализированных факторов. Таким же образом формируется множество $I_R = \{I_r\}_m$, элементами которого являются отранжированные по значимости индексы факторов, в зависимости от того какой фактор был выбран изначальным. \par
Для того, чтобы задать некоторое "глобальное" ранжирование для $\tilde{X}_R$ и $I_R$ можно воспользоваться следующим алгоритмом.

\begin{algorithm}[H]
 \KwData{$I_R = \{I_r\}_m, S = {0}_m$}
 \KwResult{Упорядоченное множество, у которого индекс элемента соответствует индексу фактора, а значение элемента - степени значимости фактора в наборе данных.}
 \For{$I_r\ \textbf{in}\ I_R$}{
    $n\_features \gets len(I_r)$\par
    \For{$i\ \textbf{in}\ I_r$}{
        $S[i] \gets n_features$\par
        $n\_features \gets n\_features - 1$\par
    }
 }
 \caption{Алгоритм ранжирования факторов по значимости}
\end{algorithm}

\newpage
\section*{Сравнение существующих подходов к ортогонализации данных в задаче регрессии}
\addcontentsline{toc}{section}{Сравнение существующих подходов к ортогонализации данных в задаче регрессии}
\hspace{0.4cm}
В рамках исследования было произвдеено сравнение ранее рассмотренных подходов для декорреляции данных на различных наборах данных с разными регрессионными моделями. В качестве метрики используется среднеквадратичная ошибка (MSE), рассчитываемая по следующей формуле: $$MSE(y, \hat{y}) = \frac{1}{|y|}(y - \hat{y})^T(y - \hat{y}).$$ Было принято решение использовать следующие широко применяемые виды решающих функций в рамках тестирования алгоритмов ортогонализации данных, \par
\begin{enumerate}
  \item Linear Regression (обычная регрессия, оцениваемая методом МНК), 
  \item Support Vector Regression (машина опорных векторов),
  \item Random Forest Regression (случайный лес).
\end{enumerate}

%вставить ссылки на hyperopt и bayesian-optimization
Перед рассчетом статистики для машины опорных векторов и случайного леса проводилась предварительная оптимизация гиперпараметров с использованием алгоритмов hyperopt и bayesian-optimization. Подробнее с их использованием можно ознакомиться на официальных страницах данных проектов. Для моделей линейной регрессии и машины опорных векторов были взяты реализации из пакета scikit-learn, соответственно, sklearn.linear. LinearRegression и sklearn.svm.SVR. Следует отметить, что в рамках программной реализации данных методов, по умолчанию, используются ралзичные методы регуляризации, которые были отключены для чистоты эксперимента. В качестве модели случайного леса использовалась реализация из пакета lightgbm.\par

Статистика представлена в виде кросс-валидации (по 8 уникальным разбиениям) по тренировочному и тестовому наборам и в виде one-fold валидации по отдельному, валидационному набору, который не пересекается с тренировочным и тестовым. Идея метода кросс-валидации заключается в том, чтобы итеративно разбивать имеющийся набор данных на два непересекающихся множества, где первое множество соответствует тренировочной выборке, а второе - тестовой, в соотношении 5 к 1. Далее выполняется <<обучение>> регрессионной модели на обучающием множестве данных и её валидация на тестовом множестве. Процедура выполняется ровно то число раз, которое соответсвует гиперпараметру алгоритма - количеству разбиений. \par

Для тестов использовались три различных набора данных, в том числе: Facebook Comment Volume Dataset, Parkinsons Telemonitoring Data Set, Energy Efficiency Data Set. Данные наборы данных достаточно популярны в рамках задач, которые включают в себя сравнение моделей машинного обучения. Это продиктовано тем, что эти наборы содержат разнообразные виды факторов (категриальные, бинарные, ординальные, а так же непрерывные), и их число сильно отличается. Помимо этого, следует отметить, что выборки очень сильно отличаются по объемам наблюдений.\par

В рамках исследования, для сравнения были отобраны следующие подходы к декорреляции данных, которые были подробно рассмотрены в предыдущих разделах работы,
\begin{enumerate}
  \item Principal Component Analysis, 
  \item Independent Component Analysis,
  \item t-Distributed Stochastic Neighbor Embedding,
  \item Locality Linear Embedding.
\end{enumerate}

Программные реализации данных алгоритмов были взяты в пакете scikit-learn. Важно отметить, что каждый из представленных алгоритмов подразумевает установку определенного, фиксированного числа независимых компонент. Из практических соображений было решено ограничиться 10-ю компонентами для PCA, ICA, LLE подходов на наборах данных Facebook Comment Volume Dataset, Parkinsons Telemonitoring Data Set. Напротив, в контексте набора данных Energy Efficency Data Set было решено ограничиться только четырьмя компонентами в связи с тем, что в этой выборке каждое наблюдение описывается только семью компонентами. Так же следует отметить, что по причине высокой вычислительной сложности алгоритма t-SNE, было решено ограничиться выделением только трех независимых компонент в каждом из наборов данных. Данное конкретное число - верхняя граница числа компонент для реализации t-SNE Барнса-Хута.

\subsection{Кросс-валидация}
\textbf{Facebook Comment Volume Dataset}

\begin{table}[H]
\begin{tabular}{llllll}
Regressor & Raw     & PCA(10)   & ICA(10) & T-SNE(3) & LLE(10) \\
LR    & 795.21  & 907.66    & 889.89  & 1218.02  & 1342.09 \\
SVR   & 1728.86 & 185357.73 & \textbf{1051.16} & \textbf{1407.01}  & \textbf{1351.36} \\
RFR   & 772.62  & 910.07    & 901.54  & 1177.28  & 1278.63
\end{tabular}
\end{table}

\textbf{Parkinsons Telemonitoring Data Set}
\begin{table}[H]
\begin{tabular}{llllll}
Regressor                 & Raw   & PCA(10) & ICA(10) & T-SNE(3) & LLE(10) \\
LR         & 6.81  & 83.11   & 86.59   & 105.74   & 113.98  \\
SVR & 17.60 & 94.73   & 88.57   & 288.82   & 114.03  \\
RFR   & 8.00  & 8.42    & 21.56   & 93.43    & 86.16  
\end{tabular}
\end{table}

\textbf{Energy Efficiency Data Set}
\begin{table}[H]
\begin{tabular}{llllll}
Regressor                 & Raw  & PCA(4) & ICA(4) & T-SNE(3) & LLE(4) \\
LR         & 6.09 & 121.46  & 94.44   & 99.63    & 110.37  \\
SVR & 9.04 & 125.52  & 99.68   & 175.71   & 110.45  \\
RFR   & 7.48 & 120.11  & 21.22   & 83.83    & 87.24  
\end{tabular}
\end{table}

\subsection{Оценка на валидационном наборе}
\textbf{Facebook Comment Volume Dataset}

\begin{table}[H]
\begin{tabular}{llllll}
Regressor & Raw     & PCA(10)   & ICA(10) & T-SNE(3) & LLE(10) \\
LR    & 914.25  & 1464.96    & 1196.91  & 1623.08  & \textbf{485.26} \\
SVR   & 275536.77 & \textbf{95049.85} & \textbf{1338.91} & \textbf{1643.11}  & \textbf{486.21} \\
RFR   & 977.35  & 1473.19    & 1153.20  & 1545.89  & \textbf{469.00}
\end{tabular}
\end{table}

\textbf{Parkinsons Telemonitoring Data Set}
\begin{table}[H]
\begin{tabular}{llllll}
Regressor                 & Raw   & PCA(10) & ICA(10) & T-SNE(3) & LLE(10) \\
LR         &  6.09  & 121.46  & 94.44   & 99.63 & 110.37  \\
SVR & 9.04 & 125.52   & 99.68   & 175.71   & 110.45  \\
RFR   & 7.48  & 120.11  & 21.22   & 83.83  & 87.24  
\end{tabular}
\end{table}

\textbf{Energy Efficiency Data Set}
\begin{table}[H]
\begin{tabular}{llllll}
Regressor                 & Raw  & PCA(4) & ICA(4) & T-SNE(3) & LLE(4) \\
LR         & 7.74 & 103.87  & 97.03   & 65.73 & 78.76 \\
SVR & 17.12 & 152.16  & 519.39  & 103.26  & 77.51  \\
RFR   & 1.34 & 107.50  & 109.79  & 14.29  & 34.39  
\end{tabular}
\end{table}

Жирным шрифтом отмечены те значения метрики MSE, которые показали некоторый прирост, полученный при использовании методов ортогонализации относительно необработанных данных. Не трудно заметить, что в большинстве случаев использование независимых компонент вместо оригинального набора данных не дало какого либо прироста. Исключительная ситуация - с набором Facebook Comment Volume Dataset, где большинство используемых методов кроме PCA показало небольшой прирост в задаче построения машины опорных векторов.

\newpage
\section*{Сравнение существующих подходов к ортогонализации данных в задаче классификации}
\addcontentsline{toc}{section}{Сравнение существующих подходов к ортогонализации данных в задаче классификации}
\hspace{0.4cm}
В данном разделе представлено сравнение ранее рассмотренных методов ортогонализации данных на различных наборах данных с разными решающими функцями. Принципиальное отличие от предыдущего раздела заключается в том, что задача регрессии была сведена к задаче классификации путем дихтомизации объясняемой величины и факторов методом бисекции по медиане на 32 части. Соответственно, в результате дихтомизации все непрерывные величины были сконвертированы в дискретные, принимающие 32 уникальных значения. В качестве метрики для оценки качества классификации предлагается использовать точность (accuracy) в связи с тем, что ее достаточно легко интерпретировать. Точность рассчитывается по следующей формуле: $$Accuracy(y, \hat{y}) = \frac{\|\{1 | y_i = \hat{y_i}\}\|}{\|y\|},$$ где выражение $\|\{1 | y_i = \hat{y_i}\}\|$ можно понимать как $TP$ (True Positive) - число верно классифицированных объектов. Поскольку решается задача множественной классификации, где моделируемая величина принимает 32 уникальных значения ($y \in \{l_1, l_2, l_3, ... l_{32}\}$), необходимо формализовать некоторый подход для экстраполяции моделей бинарной классификации на случай нескольких классов. В рамках исследования предлагается использовать подход $OvR$ (one versus rest), который заключается в построении $m$ уникальных классификаторов для $m$ отдельных классов. Каждый $i$-й классификатор решает такую бинарную проблему классификации, где $\tilde{y}_j = 1\ iif\ y_i = l_j\ else\ 0$. Некоторому объекту $x$ ставится в соответствие класс $l_i$, если $\tilde{y}_i = 1$. В рамках эксперимента были использованы следующие модели классификации: \par
\begin{enumerate}
  \item Logistic Regression (логистическая регрессия), 
  \item Support Vector Classifier (машина опорных векторов),
  \item Random Forest Classifier (классификатор случайного леса).
\end{enumerate}

Из используемых методов ортогонализации были исключены t-SNE, LLE. Это продиктовано тем, что после дихтомизации пространство факторов возрастает до $m * 32$ новых факторов. Проблема заключается в том, что данные алгоритмы крайне чувствительны к количеству используемых факторов, что чрезвычайно замедляет скорость их расчета и предъявляет высокие требования к объему оперативной памяти. В остальном, конфигурация эксперимента не отличается от той, что применялась при решении задачи построения регрессии.\par
\newpage

\subsection{Кросс-валидация}
\textbf{Facebook Comment Volume Dataset}

\begin{table}[H]
\begin{tabular}{llllll}
Regressor & Raw     & PCA(10)   & ICA(10) \\
LR    & 0.688  & 0.551    & 0.551 \\
SVC   & 0.688  & 0.551    & 0.652 \\
RFC   & 0.681  & 0.551    & 0.660
\end{tabular}
\end{table}

\textbf{Parkinsons Telemonitoring Data Set}
\begin{table}[H]
\begin{tabular}{llllll}
Regressor  & Raw   & PCA(10) & ICA(10) \\
LR         & 0.857  & 0.502  & 0.478 \\
SVC        & 0.859 & 0.505   & 0.505 \\
RFC        & 0.885  & 0.598   & 0.673  
\end{tabular}
\end{table}

\textbf{Energy Efficiency Data Set}
\begin{table}[H]
\begin{tabular}{llllll}
Regressor                 & Raw  & PCA(4) & ICA(4) \\
LR    & 0.743 & 0.730  & 0.539 \\
SVC   & 0.742 & 0.733  & 0.722 \\
RFC   & 0.759 & \textbf{0.774}  & \textbf{0.761}  
\end{tabular}
\end{table}

\subsection{Оценка на валидационном наборе}
\textbf{Facebook Comment Volume Dataset}

\begin{table}[H]
\begin{tabular}{llllll}
Regressor & Raw     & PCA(10)   & ICA(10) \\
LR    & 0.680 & 0.550 & 0.553 \\
SVC   & 0.680 & 0.550 & 0.649 \\
RFC   & 0.674 & 0.550 & 0.661
\end{tabular}
\end{table}

\textbf{Parkinsons Telemonitoring Data Set}
\begin{table}[H]
\begin{tabular}{llllll}
Regressor & Raw   & PCA(10) & ICA(10)\\
LR  & 0.841  & 0.464  & 0.464 \\
SVC & 0.846  & 0.469   & 0.488 \\
RFC & 0.877  & 0.591  & 0.656 
\end{tabular}
\end{table}

\textbf{Energy Efficiency Data Set}
\begin{table}[H]
\begin{tabular}{llllll}
Regressor & Raw  & PCA(4) & ICA(4) \\
LR  & 0.727 & 0.714  & 0.571 \\
SVC & 0.740 & 0.714  & \textbf{0.766} \\
RFC & 0.727 & \textbf{0.766}  & \textbf{0.792}
\end{tabular}
\end{table}

Жирным шрифтом отмечены те значения метрики $Accuracy$, которые показали некоторый прирост относительно необработанных данных благодаря использованию методов ортогонализации. В большинстве случаев, использование независимых компонент не отразилось положительно на результатах классификации. Однако стоит отметить, что некоторый прирост стабильно наблюдался в наборе данных Energy Efficiency Data Set для модели случайного леса. В этом случае и PCA и ICA показали стабильный прирост точности как на кросс-валидации так и на one-fold валидации с использованием обособленного валидационного набора данных.

\section*{Апробация алгоритма подбора оптимального подмножества данных}
\addcontentsline{toc}{section}{Апробация алгоритма подбора оптимального подмножества данных}
\hspace{0.4cm}
В данном разделе представлены результаты апробации алгоритма подбора оптимального подмножества на ранее рассматриваемых наборах данных, на которых проводилось тестирование алгоритмов ортогонализации, в том числе:
\begin{enumerate}
  \item Facebook Comment Volume Dataset, 
  \item Parkinsons Telemonitoring Data Set,
  \item Energy Efficiency Data Set.
\end{enumerate}
Тестирование подхода реализовывалось на задаче регрессии, сведенной к задаче классификации путем дихтомизации объясняемой величины модели и факторов методом бисекции по медиане на 32 части.
Для статистики были взяты следующие модели классификации: \par
\begin{enumerate}
  \item Logistic Regression (логистическая регрессия), 
  \item Support Vector Classifier (машина опорных векторов),
  \item Random Forest Classifier (случайный лес).
\end{enumerate}
В качестве метрики качества, как и в предыдущем разделе использовалась точность (accuracy), которая рассчитывается по следующей формуле: $Accuracy(y, \hat{y}) = \frac{\|\{1 | y_i = \hat{y_i}\}\|}{\|y\|}$, где выражение $\|\{1 | y_i = \hat{y_i}\}\|$ можно интерпретировать как $TP$ (True Positive) - число верно классифицированных объектов. \par
Далее приведена статистика, полученная с использованием кросс - валидации по 8 уникальным подразбиениям с усредненным значением метрики Accuracy, где индекс набора данных соответствует его порядковому номеру в списке, который можно найти в начале данного раздела. \par
\begin{table}[H]
\begin{tabular}{llllll}
Regressor & $dataset_1$ & $dataset_2$ & $dataset_3$ \\
LR    & 0.677 & 0.828 & 0.760 \\
SVC   & 0.678 & 0.832 & 0.759 \\
RFC   & 0.572 & 0.409 & 0.542
\end{tabular}
\end{table}

Можно обратить внимание, что в сравнении с подходами PCA и ICA, ортогонализация набора данных положительно сказалать на точности предсказаний моделей логистической регрессии и машины опорных векторов. При этом, напротив, точность классификации очень сильно падает в классификаторе случайного леса. Соответственно, основываясь на имеющихся данных, можно сделать предположение о том, что данный алгоритм эффективен на множестве линейных моделей классификации и показывает низкую точность на других подходах к классификации. Данное предположение будет верифицировано или опровергнуто в рамках последующих исследований в данной области.

\section*{Заключение}
\addcontentsline{toc}{section}{Заключение}
\hspace{0.4cm}
В рамках настоящего исследования, проведенного в целях разработки алгоритма, позволяющего выполнить ортогонализацию (декорреляцию) набора данных с использованием произвольной решающей функции, был выполнен анализ современных подходов к решению задачи ортогонализации данных, а так же их сравнение в рамках задачи регрессии и классификации. Для анализа использовались следущие наборы данных, широко применяемые в экспериментах, связанных с оценкой качества различных моделей машинного обучения,
\begin{enumerate}
  \item Facebook Comment Volume Dataset, 
  \item Parkinsons Telemonitoring Data Set,
  \item Energy Efficiency Data Set.
\end{enumerate}
Для построения регрессии были задействованы следующие подходы,
\begin{enumerate}
  \item Linear Regression (обычная регрессия, оцениваемая методом МНК), 
  \item Support Vector Regression (машина опорных векторов),
  \item Random Forest Regression (случайный лес).
\end{enumerate}
Для решения задачи классификации были использованы модели,
\begin{enumerate}
  \item Logistic Regression (логистическая регрессия), 
  \item Support Vector Classifier (машина опорных векторов),
  \item Random Forest Classifier (случайный лес).
\end{enumerate}
Сравнительный анализ показал, что в большинстве случаев ни один из использовавшихся алгоритмов ортогонализации данных не продемонстрировал прироста релевантной данной задаче метрики в сравнении с исходным, сырым набором данных. \par
Был предложен новый алгоритм для декорреляции данных и проведена его апробация на аналогичных выборках данных в рамках решения задачи классификации. Результаты показали, что новый подход эффективно выделяется на фоне остальных алгоритмов при использовании линейных моделей классификации, но испытывает деградацию в точности классификации в рамках решения поставленной задачи с использованием метода случайного леса. Было выдвинуто предположение о том, что алгоритм эффективен только при его использовании в паре с линейными моделями. В ходе дальнейших исследований планируется проверить данную гипотезу, а так же продолжить работу над доработкой нового метода декорреляции набора данных.
%\subsection{A Subsection Sample}
%Please note that the first paragraph of a section or subsection %is
%not indented. The first paragraph that follows a table, figure,
%equation etc. does not need an indent, either.
%
%Subsequent paragraphs, however, are indented.
%
%\subsubsection{Sample Heading (Third Level)} Only two levels of
%headings should be numbered. Lower level headings remain %unnumbered;
%they are formatted as run-in headings.
%
%\paragraph{Sample Heading (Fourth Level)}
%The contribution should contain no more than four levels of
%headings. Table~\ref{tab1} gives a summary of all heading %levels.
%
%\begin{table}
%\caption{Table captions should be placed above the
%tables.}\label{tab1}
%\begin{tabular}{|l|l|l|}
%\hline
%Heading level &  Example & Font size and style\\
%\hline
%Title (centered) &  {\Large\bfseries Lecture Notes} & 14 point, %bold\\
%1st-level heading &  {\large\bfseries 1 Introduction} & 12 %point, bold\\
%2nd-level heading & {\bfseries 2.1 Printing Area} & 10 point, %bold\\
%3rd-level heading & {\bfseries Run-in Heading in Bold.} Text %follows & 10 point, bold\\
%4th-level heading & {\itshape Lowest Level Heading.} Text %follows & 10 point, italic\\
%\hline
%\end{tabular}
%\end{table}


%\noindent Displayed equations are centered and set on a separate
%line.
%\begin{equation}
%x + y = z
%\end{equation}
%Please try to avoid rasterized images for line-art diagrams and
%schemas. Whenever possible, use vector graphics instead (see
%Fig.~\ref{fig1}).
%
%\begin{figure}
%\includegraphics[width=\textwidth]{fig1.eps}
%\caption{A figure caption is always placed below the illustration.
%Please note that short captions are centered, while long ones are
%justified by the macro package automatically.} \label{fig1}
%\end{figure}
%
%\begin{theorem}
%This is a sample theorem. The run-in heading is set in bold, while
%the following text appears in italics. Definitions, lemmas,
%propositions, and corollaries are styled the same way.
%\end{theorem}
%%
%% the environments 'definition', 'lemma', 'proposition', 'corollary',
%% 'remark', and 'example' are defined in the LLNCS documentclass as well.
%%
%\begin{proof}
%Proofs, examples, and remarks have the initial word in italics,
%while the following text appears in normal font.
%\end{proof}
%For citations of references, we prefer the use of square brackets
%and consecutive numbers. Citations using labels or the author/year
%convention are also acceptable. The following bibliography provides
%a sample reference list with entries for journal
%articles~\cite{ref_article1}, an LNCS chapter~\cite{ref_lncs1}, a
%book~\cite{ref_book1}, proceedings without editors~\cite{ref_proc1},
%and a homepage~\cite{ref_url1}. Multiple citations are grouped
%\cite{ref_article1,ref_lncs1,ref_book1},
%\cite{ref_article1,ref_book1,ref_proc1,ref_url1}.
%
% ---- Bibliography ----
%
% BibTeX users should specify bibliography style 'splncs04'.
% References will then be sorted and formatted in the correct style.
%
% \bibliographystyle{splncs04}
% \bibliography{mybibliography}
%
\bibliographystyle{utf8gost705u}
\bibliography{biblio}
\end{document}
